\documentclass{article}

\usepackage[utf8]{inputenc}
\usepackage[T1]{fontenc}
\usepackage{microtype}
\usepackage[polish]{babel}
\usepackage{newspaper}

\date{\today}
\currentvolume{15}

\SetPaperName{%
    \fontencoding{T1}\fontfamily{phv}\fontsize{30}{0}\bfseries
    Example Newspaper%
}

%% The name used in the running header after
%% the first page
\SetHeaderName{Example Newspaper}
\SetPaperLocation{Example Location}

\usepackage{graphicx}
\usepackage{multicol}

%% Custom package in the project directory
\usepackage{newspaper-mod}

\renewcommand{\byline}[2]{\begin {center} \bylinestyle #1 \\{\footnotesize \bf \MakeUppercase {#2}} \\ \rule [3pt]{0.4\hsize}{0.5pt}\\ \end {center} \par}

%%%%%%%%%  Front matter   %%%%%%%%%%

\usepackage{lipsum}

\begin{document}
\maketitle

\begin{multicols}{2}

\byline{Some article}{Jan Nowak}
    \lipsum[1]
\closearticle


\headline{Another headline}
This is just an example to fill up some space, but as long as I have your attention, I'll give some newspaper advice.

I suppose we could also show how an equation is type set:
\begin{displaymath}
x=\frac{-b\pm\sqrt{b^2-4ac}}{2a}
\end{displaymath}
and there you have it.  

\lipsum[1-4]

\closearticle

\byline{New page article}{Jan Kowalski}
\meaning\byline
\closearticle

\end{multicols}

\end{document}
